
% TUGboat, Vol 17, No1, March 1996

% two-coloumn style
%% incorporates editing changes by bb; 6 jan 1996

\input tugboat.sty

\input llig
\input ste

\def\q{\quad}
\long\def\okkvir#1#2#3{%
\hbox{\vrule\vbox{\hrule\vskip#1pt\hbox{\hskip#1pt\vbox{%
\hsize#2cm\parindent=0pt #3%
}\hskip#1pt}\vskip#1pt\hrule}\vrule}%
}


\font\ugv=ugve

\def\d@anger#1{\medbreak\begingroup\clubpenalty=10000
\def\par{\endgraf\endgroup\medbreak}%
\setbox0=\hbox{\ugv #1\kern2mm}%
\hangindent=\wd0
\def\slov@{\smash{\lower\baselineskip\hbox{\copy0}}}
\noindent\hangafter=-2%
\hbox to0pt{\hskip-\hangindent
\slov@\hfill}}

\outer\def\slovo{\d@anger}
\def\enddanger{\endgraf\endgroup}


\font\stechak=stechak


\title * Croatian Fonts *

\author * Darko \v Zubrini\'c *
%\address * University of Zagreb \\ FER, Avenija Vukovar 39, Zagreb \\ Croatia *
\address * University of Zagreb \\ FER, Avenija Vukovar 39 \\ Zagreb, Croatia *
\netaddress [\network{Internet}] * darko.zubrinic@fer.hr *

\article



The aim of this note is to inform the interested reader
about the possibility of obtaining several new fonts: Croatian
Glagolitic (round -- sometimes called Bulgarian glagolitic,
angular, Ba\v ska tablet, cursive,
%% bb -- re the "ba\v{s}ka plate", please see below
%ligatures, the Baromi\'c broken ligatures, caligraphic),
ligatures, the Baromi\'c broken ligatures, calligraphic),
Croatian Cyrillic, ste\'cak ornaments and Croatian wattle patterns.
They represent a considerable extension of the first version
of glagolitic fonts I created in 1992 (see [2]).

Having in mind that  Croatian Glagolitic Script has a long history of
at least 11 centuries (from 9th to 20th), it is not
surprising that  there exists a large
%variety of its handwritten and printed versions.  The author has
%created several major types of this Script.
variety of handwritten and printed versions.  The author has
created several major fonts of type using this Script.
\medskip

%\noindent$\bullet$  The so called ``round type'', together
\noindent$\bullet$  The so-called ``round type'', together
with the corresponding numerical values, is the following:

\input ttablo

\noindent In some documents there appear  additional
%versions of  letters,
versions of some letters,
%like e.g.\ the ``spider-like h'' {\obl\char'076}. It is interesting
e.g.\ the ``spider-like h'' {\obl\char'076}. The
tops of letters appearing in this font are aligned. The
``hanging style'' the round galgolitic can be illustrated
e.g.\ by
{\obl b3yzzetva}, {\obl svoem}, as it appears on the Kiiv folia
(the first folio
is from Croatia, written in region of  Dubrovnik, 11th century;
 the remaining six are from Czechia, 10th century).

\medskip

%% bb -- to me, ``plate'' often implies a flat dish meant for serving
%%       food -- these are sometimes ceremonial, and that's why i first
%%       misunderstood this sentence; i know that is not what is meant here.
%%       perhaps ``tablet'' or ``plaque'' is close enough in meaning,
%%       and would not be as confusing.
\noindent$\bullet$ The font of the Ba\v ska stone tablet
(carved in nd A.D.\
1100  on the island of Krk):

\input ttablb


The Ba\v ska tablet is one of our most important cultural
monuments  ($2\times 1$ m${}^2$). The reason is that it was
written in the Croatian
%language (with the elements of the Church Slavonic) as early as
language (with elements of Church Slavonic) as early as
in the 11th century.  Its text comprises more than $400$
letters and  contains the earliest
mention of a Croatian king written in the Croatian
vernacular: Z''v''nimir'', kral'' hr''vat''sk''\"{\i}
(Hrvatska $=$ Croatia), i.e.\
{\it Zvonimir, the Croatian king\/}, or in the
Glagolitic:
\setbox0=\hbox{\bassv kral3 \ hr3vxt3sk3y}
$$
\okkvir36{\hbox to \wd0{\hss\bassv z3v3nimir3\hss}
\vskip2mm
\hbox{\bassv kral3 \ hr3vxt3sk3y}}
$$
There are also earlier  monuments from the 9th century that
mention Croatian kings and dukes, but written in the Latin
Script and in the Latin
language.

%The reader will notice that several Latin and(or) Cyrillic
The reader will notice that several Latin and/or Cyrillic
letters appear on the table: {\bass O, I, M, N, T, V} ($=$
%V). This
V)\null. This
is only one among numerous proofs of the parallel use of three
%Scripts (Glagolitic, Latin, Cyrillic) and three languages
scripts (Glagolitic, Latin, Cyrillic) and three languages
(Croatian, Church Slavonic and Latin) in Croatia. The
%three-scriptural and three-language character of the
three-script and three-language character of the
Croatian Middle Ages is a unique phenomenon in the history of
European culture (see [1]).

\medskip

\noindent$\bullet$ Since the 12th century the Glagolitic
%Script survived only on the Croatian soil. Until that time
Script has survived only in Croatia. Until that time
%it existed also in some other regions, like in Bulgaria,
it existed also in some other regions, e.g.\ Bulgaria,
Macedonia, Roumania,   Ukraine, when it was replaced by
%theillic Script. In Croatia there developped the so
%called {\it angular form\/} of the Glagolitic.
the Cyrillic Script, and also in Czechia. In Croatia there developed the so-called
{\it angular form\/} of the Glagolitic.

\input ttabl

\noindent Its  golden period
falls between  the 12th and the 16th century:
After that a
%decline of  this Script ensued, as a result of
decline of  this script ensued, as a result of
the penetration of the Ottoman Empire.
\medskip


\noindent$\bullet$ The Croatian Glagolitic has hundreds of interesting
ligatures.
 Let us present some of them:

\input ttlis


\noindent It is
striking that a {\it printed\/} Glagolitic book, the Brozi\'c breviary from 1561
(1081 pages),  has as many as $250$ ligatures.


\noindent$\bullet$ A unique
creation in the history of European printing are the so called
{\it Baromi\'c broken ligatures\/}. The idea was to add one half of
a letter (say {\lom A} of {\mgl A}) to another (say {\mgl
B}), to obtain a broken ligature (\hbox{{\mgl B}{\lom A}} -- ba). Other
combinations are also possible with
{\lom D} ({\mgl D}, D),
{\lom ZZ} ({\mgl ZZ}, \v Z),
%{\lom L} ({\mgl L}, T),
{\lom L} ({\mgl L}, L),
{\lom V} ({\mgl V}, V),
{\lom T} ({\mgl T}, T).
Broken ligatures were used in the Baromi\'c Missal, the
incunabula printed in the Croatian city of Senj in 1494
(last year we celebrated its $500$th
anniversary). Only three samples are preserved: one in
the Saltyko S\v cedrin Library in St.Petersburg (Russia),
one and the only complete copy in
the Szeczenyi Library in Budapest (Hungary), and one
in Croatia on the island of Cres. When looking at the text
containing Baromi\'c broken ligatures,  one has the impression as if this
incunabula was handwritten. Few examples: \hbox{{\mgl B}{\lom
A}{\mgl ROMICH}} (Baromi\'c), \hbox{{\mgl D}{\lom AZZD6}} (rain), \hbox{\mgl
MO{\lom L}{\mgl I}{\lom TVA}} (prayer).


%Hundreds of Croatian Glagolitic monuments, both handwritten and
Hundreds of Croatian Glagolitic texts, both handwritten and
printed, the oldest from the 12th century,
 are held in national museums in more than $20$ European
countries, and also in the USA.\footnote{I would like to
take the opportunity to send an appeal to those readers who maybe
know of any of the Croatian Glagolitic documents
held in private possession to inform me.}

% The reader living in New York can see the beuatiful
 The reader living in New York can see the beautiful
Croatian Missal from around 1410 in the Pierpont Morgan
Library (reprinted by the Martin Sagner Verlag, Munich, 1976). In
%the Congress Library in Washington you can see a sample of
the Library of Congress in Washington you can see a sample of
the first Croatian incunabula, printed in  1483
(unfortunately, it is not known where precisely).
Six copies are in Croatia, two in the Vatican Library, one
in the National Library in Vienna (Austria) and one  in the
Saltykov \v S\v cedrin
Library in St.Petersburg (Russia). It was the first book in
the history of European printing that was not printed in the
Latin characters, as well as the first incunabula not printed in the
Latin language.

Probably the most valuable
Croatian Glagolitic book is the Missal of Hrvoje (1404), held in
the Library of Turkish sultans (Topkapi Saray) in Constantinople.



 It is interesting that a Croat George (Juraj) de Slavonie, or de
Sorbonne (14/15th century),
a professor at Sorbonne in
Paris,
left us several valuable Glagolitic manuscripts
written by his hand, held today in
Municipal Library in Reims (France). One of them contains a prayer
``Our Father'', on which we would like to illustrate the Croatian
Glagolitic:


\slovo{O}{\mgl cce nass6 izze esi nanebesih6 * sveti se ime
\tvo e * {\gg\char'005}idi cesars\tv o \tv oe * budi vola \tv oya yako
nanebesi inaze\ml i * \lower1.33pt\hbox{H}lib nass vsedanni dai
nam'ga danas i \ot pusti nam' dl'gi nasse *
\lower1.33pt\hbox{Ya}kozze i mi \ot pusschaem'  dl'ge dlzznikom
nassim6 * i nevavedi nas' v' napast' * \lower1.33pt\hbox{N}a
iz'bavinas' od ne{\gg\char'005}iyazni} % \char'005=pr

\medskip

%\noindent$\bullet$ In the 16th and 17th century there began
%to appear some caligraphic Glagolitic letters in Crotian
\noindent$\bullet$ In the 16th and 17th century
some calligraphic Glagolitic letters began to appear in Croatian
printed books:
$$
\gather
\text{
{\kal A} (A),
{\kal V} (V),
{\kal D} (D),
{\kal Z} (Z),
{\kal I} (I),}\\
\text{
{\kal K} (K),
{\kal L} (L),
{\kal M} (M),
{\kal N} (N),
{\kal P} (P),}\\
\text{
{\kal R} (R),
{\kal S} (S),
{\kal H} (H),
{\kal CC} (\v C).}
\endgather
$$
\medskip

\noindent$\bullet$ There are thousands of
{\it cursive\/} Glagolitic
documents, witnessing above all about the Croatian language
and its very early use in official documents, and about
 highly organized civil life
 in the Middle Ages. Probably the
%most important is the {\it Vinodole Code\/} from  1288. This
most important is the {\it Vinodol Code\/} from  1288.
%% bb -- this is the spelling used above
  Very important
is the {\it Istarski razvod\/} from
the 14th century, written in  the region of Istria in three
official copies:
in the Latin and German languages (in Latin Script), and in
the
Croatian language (as was expressly stated), using the
Glagolitic Script. It defined the borders between different
rulers in Istria.
Equally important are the city
statutes of many Croatian cities written in the Glagolitic, the earliest  dating  from
the 14th century. Here is
a variant of the cursive Croatian Glagolitic
 (we provide also the angular type for comparison):

\input ttablokurz

\noindent An example: {\kur Darko Zzubrinich}.

\medskip

\noindent$\bullet$ Croatian Cyrillic (also called Bosan\v
cica or Bosanica)
 was quite widespread
%in Bosnia and  in the Dalmatian part of Croatia. Its developement
among the Croats in Bosnia and in the Dalmatian part of Croatia.
Its development
can be traced from the 12th to 19th century.
Here it is:

\input ttablhc


%\noindent Probably the most beutiful Croatian Cyrillic book
\noindent Probably the most beautiful Croatian Cyrillic book
is the Missal of Hval written in 1400-1404, now held in the University
Library in Bologna (Italy), reprinted in Sarajevo in 1986.
\medskip


\noindent$\bullet$ There
%exist more that $66,000$ mysterious tombstone monuments, mostly in
exist more than $66,000$ mysterious tombstone monuments, mostly in
%Bosnia-Herzegovina and  Croatia, called  {\it ste\'cak\/},
Bosnia-Herzegovina and  Croatia, called  {\it ste\'cak\/}
 (13-14th century),
some of them having short engravings in the
Croatian Cyrillic Script (e.g.\ ``please do not
disturb me, I was like you and
you will be like me''), with  interesting and simple border
%decorations, like:
decorations, like these:
\medskip
\cvijet
\smallskip
\kukagd
\smallskip
\dvije8
\smallskip
\Oo

\medskip
\noindent  dancers:
$$
\text{\stechak lkkkkkd}
$$
human-like figures:
$$
\text{\stechak L\q L\q L}
$$
birds:
$$
\text{\stechak p q}
$$
star-like ornaments:
$$
\gather
\text{\stechak 4\q Z}\\
\text{\stechak a\q b\q 5}\\
\text{\stechak i\q j\q e\q f}
\endgather
$$
a circle (or a wheel),  a symbol of eternal life:
$$
\text{\stechak g\q h\q K}
$$
swastika:
$$
\text{\stechak s}
$$
various crosses:
$$
\text{\stechak
A\q B\q C\q D\q T}
$$
The following ornament is frequent:
$$
\text{\stechak R P a}\hbox to0pt{\,.\hss}
$$
%\noindent These monuments belong to Krstyans, memebers of the Bosnian
\noindent These monuments belong to Krstyans, members of the Bosnian
Church, a Christian religious sect about which we still know very
little.


\noindent$\bullet$ Let me finish this article with some of
the most typical Croatian wattle patterns appearing in
 our preromanesque churches, built between the  9th and 12th
 century. From
 about $300$ preromanesque Croatian
 churches only  $15$ are well preserved. The most widespread
 wattle pattern is
$$
\text{\pleter 1eeeeeeeeeeeeeeeeee2}
$$
\noindent Of course, it is composed of {\pleter 1 e e 2}.
Somewhat more complex
patterns are:
%\centerline{\plet ommp}
$$
\gather
\text{\ple a}\\
\vbox to1mm{}\\
\text{\plet srrt}
\endgather
$$

%\noindent Additional informations about the Croatian Glagolitic can be
\noindent Additional information about Croatian Glagolitic can be
seen at the URL:
$$
\text{\tt http://www.tel.fer.hr/hrvatska/Croatia-HCS.html}
$$
%(see ``Croatia -- its History, Culture and Science'').
All  the fonts appearing in this article will be available
 freely  via the WWW on  the CTAN web.
%see {\tt  croatian}).
All ligatures and fonts are defined in the file {\tt llig.tex} of the
package.
 The angular glagolitic font is activated by
{\tt$\backslash$mgl} (a slightly larger version can be obtained using
{\tt$\backslash$ngl}).
If one wants to use glagolitic
ligatures, the definition list in {\tt llig.tex} should be
consulted. For example, to obtain {\mgl\ml} (ml) one has to write
{\tt$\backslash$mgl$\backslash$ml}. Other fonts are
activated in a similar way (they contain no glagolitic
ligatures): {\tt$\backslash$obl} (round glagolitic,
{\tt$\backslash$nobl} is a bit larger),
{\tt$\backslash$bass} (Baska tablet),
{\tt$\backslash$kur} (cursive),
{\tt$\backslash$kal} (caligraphic),
{\tt$\backslash$lom} (Baromich broken ligatures)
{\tt$\backslash$hc} (Croatian Cyrillic),
and several wattle
patterns are activated by
{\tt$\backslash$ple},
{\tt$\backslash$plet},
{\tt$\backslash$pleter}.
In general, glagolitic letters corresponding to {\it \v c, \v z,
\v s, \'c (\v s\'c, \v st)} are encoded in \TeX\ ligtables as {\tt cc,
zz, ss, ch} respectively. More precisely, \hbox{{\mgl cc}} is
obtained from {\tt$\backslash$mgl cc}.
Capital letters are obtained by using CC (or Cc) etc.
Using {\tt cx} you can obtain yet another version of the
angular \v c: \hbox{{\mgl cx}}.
The so-called `djerv' is obtained by typing  {\tt j} (or {\tt
J}) in the
round glagolitic. In the angular form the following two version of the
`djerv' can be obtained:
  {\mgl j} ({\tt j}) and \hbox{{\mgl dd}} ({\tt dd}).
  `Yat' -- \hbox{\mobl ya}, \hbox{\mgl ya} -- appearing in round and angular
  glagolitic  can be obtained using {\tt ya} (or {\tt YA,
  Ya}).
  Moreover, if `yat' is to be read as `ye' (which is the
  case when it
  appears after a consonant), then you can use {\tt ye} as
  well.
  The letter `yu' {\mgl yu} is obtained by typing {\tt yu},
  `izze' ({\obl y}, {\mgl y}) by typing {\tt y}.
  The same for the Croatian cyrillic. Two versions of
   very frequently used   {\it Croatian semivowels\/} can be obtained:
  either  by typing apostrophe {\tt '} (`yerok'; for example
  in {\mgl bog'} - God) or
  {\tt6} (`yer'; {\mgl bog6}). Definitions  of semivowels
  `yer' and `yor' and other letters (like
{\mobl\char'003,  \char'005}) appearing in some oldest Croatian and Bulgarian --
  Macedonian  documents written in the round glagolitic
  can be seen in  metafont files {\tt oblm.mf} and {\tt
  oblv.mf}, that are inputted  in {\tt obl.mf} - a round glagolitic
  metafont file.

As you see, an appropriate
hypenation table is necessary to prevent line breaking
between  pairs of letters like  {\tt cc, zz, ss, ch, dd,
ya, ye,} and it has been
included to the Croatian font package too ({\tt
glhypehn.tex}). The table also
incorporates some basic
hypenation rules of the Croatian language.
Definition
names of  many other symbols appearing in various fonts can be seen
by looking into  source files accompaning  this text.

%In the case you use Croatian fonts, I would deeply appreciate
%to inform me.
 If you should use Croatian fonts, I would deeply appreciate
if you would inform me.



\medskip

\noindent{\bf References}


\item{[1]} Eduard Hercigonja: {\it Tropismena i trojezi\v cna
kultura hrvatskoga srednjovjekovlja\/} (Three-script and
three-language culture of the Croatian Middle Ages), Matica
hrvatska,  Zagreb, 1994 (in Croatian),

\item{[2]} Darko \v Zubrini\'c: {\it The exotic Croatian
Glagolitic Alphabet\/}, TUGboat, Vol 13, No 4, 1992, p.\
470--471.


\makesignature
\endarticle

