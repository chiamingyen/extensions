% Self-documenting file

\ifx  \Comment \Undefined
      \long \def \Comment #1\EndComment {}
\else
      \ifx  \EndComment \Undefined
            \def \EndComment
                {\message {Error: \noexpand \EndComment encountered
                                without preceding \noexpand \Comment}}
      \else %%% both defined; let's hope they're the right definitions!
      \fi
\fi

%%% Author:  Philip Taylor (RHBNC) <P.Taylor@Rhul.Ac.Uk>
%%% Cropmarks.TeX V1.4 Copyright (C) Philip Taylor, MCMLXXXXII

\Comment % The Licence

 Unlimited copying and redistribution of this file
 is permitted so long as the file is not modified
 in any way.

 Modifications may be made for private purposes (though
 this is discouraged, as it could result in documents
 typesetting differently on different systems) but if
 such modifications are re-distributed, the modified
 file must not be capable of being confused with the
 original.  In particular, this means

(a) the filename (the portion before the extension, if any)
    must not match either of the sequences :

        CROPMARKS
        CROP-MARKS

   regardless of case, and

(b) the file must contain conditions identical to these,
 except that the modifier/distributor may, if he or she
 wishes, augment the list of proscribed filenames.

\EndComment

%%% Usage:

%%% \input cropmarks
%%% \topcropmark = <dimen> [plus <dimen>] [minus <dimen>]
%%% \bottomcropmark = <dimen> [plus <dimen>] [minus <dimen>]
%%% \leftcropmark = <dimen> [plus <dimen>] [minus <dimen>]
%%% \rightcropmark = <dimen> [plus <dimen>] [minus <dimen>]
%%% \lrcropmarkstrue ;;; iff you want alternating left and right cropmarks only
%%% \rvcropmarkstrue ;;; iff you want recto and verso specified independently
%%%                     (in which case the first four variables are two-valued
%%%                      as \recto<var> and \verso<var>).

%%% Square brackets indicate optional items, and the <dimen>s are
%%% respectively (and all measured w.r.t. the origin, which is the
%%% top left corner of the notional, rather than physical page; the
%%% exact origin will need to be determined by inspection)

%%%     The distance to the centre of the cross;
%%%     The distance which the cross extends beyond the page limits;
%%%     The distance which the cross extends within the page limits.

%%% Any or all can be zero <unit>s.  For conventional (exo-) cropmarks, the
%%% `plus' component indicates the length of the mark, and the `minus' component
%%% should be omitted or zero.  For inverted (endo-) cropmarks, the `plus'
%%% component should be omitted or zero, and the `minus' component indicates the
%%% length of the mark.  If `plus' or `minus' components are negative, this
%%% has the effect of a `white cropmark' extending in the reverse direction
%%% for the specified distance; this may be used to achieve non-touching marks.
%%% The first and only mandatory component may be also negative, to cope with
%%% pathological origins.

%
%%% A test which will allow us to be read regardless of the \catcode of `@'
%

\newif \ifatletter
\ifcat @A\atlettertrue \else \catcode `\@ = 11 \atletterfalse \fi

\ifx \c@tcodeminus \undefined \else \message {Warning: previous
meaning of \string \c@tcodeminus \space lost} \fi \edef
\c@tcodeminus {\the \catcode `\-} \catcode `\- = 12

%
\message {CROPMARKS.TEX V1.4 Copyright (C) 05-Mar-1992 13:43:28
Philip Taylor}
%

%
%%% Miscellaneous variables
%
\newif    \iflrcropmarks            \lrcropmarksfalse
\newif    \ifrvcropmarks                \rvcropmarksfalse
\newdimen \cropmarkwidth            \cropmarkwidth = 0.4 pt
\newdimen \t@pshift
\newdimen \b@ttomshift
\newdimen \l@ftshift
\newskip  \topcropmark
\newskip  \leftcropmark
\newskip  \rightcropmark
\newskip  \bottomcropmark
\newskip  \rectotopcropmark
\newskip  \rectoleftcropmark
\newskip  \rectorightcropmark
\newskip  \rectobottomcropmark
\newskip  \versotopcropmark
\newskip  \versoleftcropmark
\newskip  \versorightcropmark
\newskip  \versobottomcropmark
\newskip  \m@rkskip
\newskip  \m@rkplus
\newskip  \m@rkminus
\newbox   \p@gebox
\newbox   \cr@pmarkbox
\newbox   \topcr@pmarkbox
\newbox   \bottomcr@pmarkbox
%
%%% Extraordinary arcanery, to parse <skip>s into <dimen>s; any better ways ?
%
\begingroup
\catcode `\P = 12 \catcode `\L = 12 \catcode `\U = 12 \catcode `\S =
12 \catcode `\M = 12 \catcode `\I = 12 \catcode `\N = 12 \catcode
`\T = 12 \globaldefs = 1 \lowercase
{%
\def \p@rseskip #1%
    {\expandafter \p@rs@skip \the #1 PLUS 0.0PT MINUS 0.0PT \@ndparse #1 }
\def \p@rs@skip #1 PLUS #2 MINUS #3\@ndparse #4
    {%
     \m@rkskip  = #1%
     \m@rkplus  = #2%
     \m@rkminus = #3%
     \ifdim \m@rkminus = 0.0 pt
        \expandafter \p@rs@sk@p \the \m@rkskip MINUS 0.0PT \@ndparse
     \fi
    }%
\def \p@rs@sk@p #1MINUS #2\@ndparse {\m@rkskip = #1 \m@rkminus = #2 }
}
\endgroup
%
%%% Take control of \shipout, to force it to add cropmarks
%
\let \ship@ut = \shipout
\def \shipout {\futurelet \b@x \whichb@x}
\def \whichb@x
    {%
     \ifx   \b@x \vbox
        \def \n@xt {\setbox \p@gebox = }%
            \afterassignment \Shipout
     \else  \ifx    \b@x \hbox
            \def \n@xt {\setbox \p@gebox = }%
                \afterassignment \Shipout
        \else   \ifx    \b@x \box
                \def \n@xt {\setbox \p@gebox = }%
                \afterassignment \Ship@ut
            \else   \ifx    \b@x \copy
                    \def \n@xt {\setbox \p@gebox = }%
                    \afterassignment \Ship@ut
                \else
                    \def \n@xt {\ship@ut}%
                    \message {Sorry, can't understand
                            what follows \shipout}%
                \fi
            \fi
        \fi
     \fi
     \n@xt
    }
\def \Shipout {\aftergroup \Ship@ut}
\def \Ship@ut
    {%
         \ifrvcropmarks
                \ifodd \pageno
                        \topcropmark = \rectotopcropmark
                        \bottomcropmark = \rectobottomcropmark
                        \leftcropmark = \rectoleftcropmark
                        \rightcropmark = \rectorightcropmark
                \else
                        \topcropmark = \versotopcropmark
                        \bottomcropmark = \versobottomcropmark
                        \leftcropmark = \versoleftcropmark
                        \rightcropmark = \versorightcropmark
                \fi
         \fi
     \setbox \topcr@pmarkbox =
     \hbox   \bgroup
         \p@rseskip \leftcropmark
         \iflrcropmarks
            \ifodd \count 0
            \let \next = \phantom
            \else
            \let \next = \relax
            \fi
         \else
            \let \next = \relax
         \fi
         \next {\ifdim  \m@rkminus < 0 pt
                                \begingroup
                                \advance \m@rkplus by \m@rkminus
                    \vrule height \cropmarkwidth
                                       depth 0 pt width \m@rkplus
                                \kern -\m@rkminus
                                \endgroup
                        \else
                    \vrule height \cropmarkwidth
                                        depth 0 pt width \m@rkplus
                        \fi
                    \begingroup
                    \p@rseskip \topcropmark
                    \vrule  height \m@rkplus
                        depth \m@rkminus
                        width \cropmarkwidth
                    \endgroup
                        \ifdim  \m@rkplus < 0 pt
                                \begingroup
                                \advance \m@rkminus by \m@rkplus
                    \rlap {\kern -\m@rkplus
                                       \vrule height \cropmarkwidth
                          depth 0 pt width \m@rkminus}%
                                \endgroup
                        \else
                    \rlap {\vrule height \cropmarkwidth
                          depth 0 pt width \m@rkminus}%
                        \fi
               }%
         \hskip \m@rkskip
         \p@rseskip \rightcropmark
         \hskip \m@rkskip
         \iflrcropmarks
            \ifodd \count 0
            \let \next = \relax
            \else
            \let \next = \phantom
            \fi
         \else
            \let \next = \relax
         \fi
         \next {\ifdim  \m@rkplus < 0 pt
                                \begingroup
                                \advance \m@rkminus by \m@rkplus
                    \llap {\vrule   height \cropmarkwidth
                    depth 0 pt width \m@rkminus
                                        \kern -\m@rkplus
                                      }%
                                \endgroup
                        \else
                    \llap {\vrule   height \cropmarkwidth
                    depth 0 pt width \m@rkminus
                                      }%
                        \fi
                                \begingroup
                \p@rseskip \topcropmark
                \vrule  height \m@rkplus
                        depth \m@rkminus
                    width \cropmarkwidth
                \endgroup
                        \ifdim  \m@rkminus < 0 pt
                                \begingroup
                                \advance \m@rkplus by \m@rkminus
                                \kern -\m@rkminus
                    \vrule height \cropmarkwidth
                                depth 0 pt width \m@rkplus
                                \endgroup
                        \else
                    \vrule height \cropmarkwidth
                                       depth 0 pt width \m@rkplus
                        \fi
               }%%%%%%%% end of parameter to \next
         \egroup%%%%%%%% end of \bgroup ... \egroup to \setbox ...
     \ht \topcr@pmarkbox = 0 pt
     \dp \topcr@pmarkbox = 0 pt
     \wd \topcr@pmarkbox = \wd \p@gebox
     \setbox \bottomcr@pmarkbox =
     \hbox   \bgroup
         \p@rseskip \leftcropmark
         \iflrcropmarks
            \ifodd \count 0
            \let \next = \phantom
            \else
            \let \next = \relax
            \fi
         \else
            \let \next = \relax
         \fi
         \next {\ifdim  \m@rkminus < 0 pt
                                \begingroup
                                \advance \m@rkplus by \m@rkminus
                    \vrule height \cropmarkwidth
                                       depth 0 pt width \m@rkplus
                                \kern -\m@rkminus
                                \endgroup
                        \else
                    \vrule height \cropmarkwidth
                                        depth 0 pt width \m@rkplus
                        \fi
                    \begingroup
                    \p@rseskip \bottomcropmark
                    \vrule  height \m@rkminus
                        depth \m@rkplus
                        width \cropmarkwidth
                    \endgroup
                        \ifdim  \m@rkplus < 0 pt
                                \begingroup
                                \advance \m@rkminus by \m@rkplus
                    \rlap {\kern -\m@rkplus
                                       \vrule height \cropmarkwidth
                          depth 0 pt width \m@rkminus}%
                                \endgroup
                        \else
                    \rlap {\vrule height \cropmarkwidth
                          depth 0 pt width \m@rkminus}%
                        \fi
               }%
         \hskip \m@rkskip
         \p@rseskip \rightcropmark
         \hskip \m@rkskip
         \iflrcropmarks
            \ifodd \count 0
            \let \next = \relax
            \else
            \let \next = \phantom
            \fi
         \else
            \let \next = \relax
         \fi
         \next {\ifdim  \m@rkplus < 0 pt
                                \begingroup
                                \advance \m@rkminus by \m@rkplus
                    \llap {\vrule   height \cropmarkwidth
                    depth 0 pt width \m@rkminus
                                        \kern -\m@rkplus
                                      }%
                                \endgroup
                        \else
                    \llap {\vrule   height \cropmarkwidth
                    depth 0 pt width \m@rkminus
                                      }%
                        \fi
                                \begingroup
                \p@rseskip \bottomcropmark
                \vrule  height \m@rkminus
                        depth \m@rkplus
                    width \cropmarkwidth
                \endgroup
                        \ifdim  \m@rkminus < 0 pt
                                \begingroup
                                \advance \m@rkplus by \m@rkminus
                                \kern -\m@rkminus
                    \vrule height \cropmarkwidth
                                depth 0 pt width \m@rkplus
                                \endgroup
                        \else
                    \vrule height \cropmarkwidth
                                       depth 0 pt width \m@rkplus
                        \fi
               }%%%%%%%% end of parameter to \next
         \egroup%%%%%%%% end of \bgroup ... \egroup to \setbox ...
     \ht \bottomcr@pmarkbox = 0 pt
     \dp \bottomcr@pmarkbox = 0 pt
     \wd \bottomcr@pmarkbox = \wd \p@gebox
     \setbox \cr@pmarkbox = \vbox
        \bgroup
        \offinterlineskip
        \p@rseskip \leftcropmark    \l@ftshift = \m@rkskip
        \p@rseskip \topcropmark     \t@pshift = \m@rkskip
        \p@rseskip \bottomcropmark  \b@ttomshift = \m@rkskip
        \advance \l@ftshift by \m@rkplus
        \advance \l@ftshift by \cropmarkwidth
        \vskip -\t@pshift
            \moveleft \l@ftshift \box \topcr@pmarkbox
        \vskip +\t@pshift
        \vskip +\b@ttomshift
            \moveleft \l@ftshift \box \bottomcr@pmarkbox
        \vskip -\b@ttomshift
        \egroup
     \setbox \p@gebox = \vbox
        {\offinterlineskip
         \box \cr@pmarkbox
         \box \p@gebox}%
     \ship@ut \box \p@gebox
    }
%
%%% Restore \catcode of `-' & `@'
%

\catcode `\- = \c@tcodeminus
\let \c@tcodeminus = \undefined
\ifatletter \else \catcode `\@ = 12 \fi

%%% All done
